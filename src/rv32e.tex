\chapter{RV32E和RV64E基础整数指令集(1.95版本)}
\label{rv32e}

这章描述了一个RV32E和RV64E基础整数指令集的建议草案,它们是为嵌入式系统的微控制器设计的。
RV32E和RV64E分别是RV32I和RV64I的简化版本:
仅有的改变是把整数寄存器的数目减少到了16个。
这章仅仅概述了RV32E/RV64E和RV32I/RV64I之间的不同,并因此应当被放在第2章~\ref{rv32}和第7章\ref{rv64}之后阅读。
% This chapter describes a proposal for the RV32E and RV64E base integer
% instruction sets, designed for microcontrollers in embedded systems.
% RV32E and RV64E are reduced versions of RV32I and RV64I, respectively:
% the only change is to reduce the number of integer registers to 16.
% This chapter only outlines the differences between RV32E/RV64E and
% RV32I/RV64I, and so should be read after Chapters~\ref{rv32} and
% \ref{rv64}.

\begin{commentary}
RV32E被设计为,为嵌入式微控制器提供一个更小的基础核心。
RV64E也有兴趣用于大型SoC设计中的微控制器、以及减少高线程(highly thread)64位处理器的上下文状态。
% RV32E was designed to provide an even smaller base core for embedded
% microcontrollers.  There is also interest in RV64E for
% microcontrollers within large SoC designs, and to reduce context state
% for highly threaded 64-bit processors.

RV32E和RV64E可以与所有当前的标准扩展相组合。
% RV32E and RV64E can be combined with all current standard extensions.
\end{commentary}

\section{RV32E和RV64E编程模型}

RV32E和RV64E把整数寄存器的数目减少到16个通用目的寄存器,({\tt x0}--{\tt x15}),这里{\tt x0}是一个专用的零寄存器。
% RV32E and RV64E reduce the integer register count to 16
% general-purpose registers, ({\tt x0}--{\tt x15}), where {\tt x0} is a
% dedicated zero register.

\begin{commentary}
我们已经发现,在小型RV32I内核设计中,较高的16个寄存器消费了除内存外的所有内核区域的大约四分之一,
因此它们的移除节省了大约25\%的内存区域,而内核的电量也相应地减少了。
% We have found that in the small RV32I core implementations, the upper
% 16 registers consume around one quarter of the total area of the core
% excluding memories, thus their removal saves around 25\% core area
% with a corresponding core power reduction.
\end{commentary}

\section{RV32E和RV64E指令集编码}

RV32E及RV64E分别使用与RV32I和RV64I相同的指令集编码,但是只提供寄存器{\tt x0} - {\tt x15}。
所有指定其它寄存器{\tt x16} - {\tt x31}的编码都是保留的
% RV32E and RV64E use the same instruction-set encoding as RV32I and
% RV64I respectively, except that only registers {\tt x0}--{\tt x15} are
% provided.  All encodings specifying the other registers {\tt x16}--{\tt
%   x31} are reserved.

\begin{commentary}
本章的前一稿将所有使用{\tt x16} - {\tt x31}寄存器的编码都可用于自定义的。
这一版本采用了一种更加保守的方法,让这些编码被保留,以便以后可以在自定义空间或新的标准编码之间分配它们。
% The previous draft of this chapter made all encodings using the {\tt
%   x16}--{\tt x31} registers available as custom.  This version takes a
% more conservative approach, making these reserved so that they can be
% allocated between custom space or new standard encodings at a later
% date.
\end{commentary}
