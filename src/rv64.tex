\chapter{RV64I基础整数指令集(2.1版本)}
\label{rv64}

这章描述了RV64I基础整数指令集,它是在第~\ref{rv32}章中描述的RV32I变体之上构建的。
这章只呈现了与RV32I的不同,所以应当与那篇更早的章节结合着阅读。
% This chapter describes the RV64I base integer instruction set, which
% builds upon the RV32I variant described in Chapter~\ref{rv32}.  This
% chapter presents only the differences with RV32I, so should be read in
% conjunction with the earlier chapter.

\section{寄存器状态}

RV64I把整数寄存器和所支持的用户地址空间拓宽到64位(表~\ref{gprs}中XLEN=64)。
% RV64I widens the integer registers and supported user address space to
% 64 bits (XLEN=64 in Figure~\ref{gprs}).

\section{整数运算指令}

大多数整数运算指令在XLEN位的值上操作。在RV64I中提供了额外的指令变体来操作32位的值,通过在操作码上添加“W”后缀来表示。
这些“*W”指令忽略了它们的输入的高32位,并且总是产生32位有符号的值、把它们符号扩展到64位,也就是说,从位XLEN-1位到位31是相等的。
% Most integer computational instructions operate on XLEN-bit values.
% Additional instruction variants are provided to manipulate 32-bit
% values in RV64I, indicated by a `W' suffix to the opcode.  These
% ``*W'' instructions ignore the upper 32 bits of their inputs and
% always produce 32-bit signed values, sign-extending them to 64 bits,
% i.e. bits XLEN-1 through 31 are equal.

\begin{commentary}
编译器和调用约定维持了一种不变性,即在64位寄存器中,所有的32位值都以一种符号扩展的格式被保持。
甚至32位无符号整数也会把位31扩展到位63~32。
因此,在无符号32位整数和有符号32位整数之间的转换是一个no-op,从一个有符号32位整数转换到一个有符号64位整数也是如此。
在这种不变性下,现有的64位宽SLTU和无符号分支比较仍然能正确地操作无符号32位整数。
类似地,现有的在32位符号扩展整数上的64位宽逻辑操作保留了符号扩展属性。加法和移位需要少量的新指令(ADD[I]W/SUBW/SxxW),
以确保32位值的合理的性能。
% The compiler and calling convention maintain an invariant that all 32-bit
% values are held in a sign-extended format in 64-bit registers.  Even 32-bit
% unsigned integers extend bit 31 into bits 63 through 32.  Consequently,
% conversion between unsigned and signed 32-bit integers is a no-op,
% as is conversion from a signed 32-bit integer to a signed 64-bit
% integer.  Existing 64-bit wide SLTU and unsigned branch compares still operate
% correctly on unsigned 32-bit integers under this invariant.  Similarly,
% existing 64-bit wide logical operations on 32-bit sign-extended integers
% preserve the sign-extension property.  A few new instructions
% (ADD[I]W/SUBW/SxxW) are required for addition and shifts to ensure reasonable
% performance for 32-bit values.
\end{commentary}

\newpage
\subsubsection*{整数寄存器 - 立即数指令}
\vspace{-0.4in}
\begin{center}
\begin{tabular}{M@{}R@{}S@{}R@{}O}
\\
\instbitrange{31}{20} &
\instbitrange{19}{15} &
\instbitrange{14}{12} &
\instbitrange{11}{7} &
\instbitrange{6}{0} \\
\hline
\multicolumn{1}{|c|}{imm[11:0]} &
\multicolumn{1}{c|}{rs1} &
\multicolumn{1}{c|}{funct3} &
\multicolumn{1}{c|}{rd} &
\multicolumn{1}{c|}{opcode} \\
\hline
12 & 5 & 3 & 5 & 7 \\
I-立即数[11:0] & src & ADDIW  & dest & OP-IMM-32 \\
\end{tabular}
\end{center}

ADDIW是一个RV64I指令,它把符号扩展的12位立即数加到寄存器{\em rs1},并在{\em rd}中产生合适的32位符号扩展的结果。
运算结果的低32位符号扩展到64位作为结果,而忽略了溢出。
注意,{\em rd, rs1, 0}把寄存器{\em rs1}的低32位的符号扩展写入寄存器{\em rd}(汇编器伪指令SEXT.W)。
% ADDIW is an RV64I instruction that adds the sign-extended 12-bit
% immediate to register {\em rs1} and produces the proper sign-extension
% of a 32-bit result in {\em rd}.  Overflows are ignored and the result
% is the low 32 bits of the result sign-extended to 64 bits.  Note,
% ADDIW {\em rd, rs1, 0} writes the sign-extension of the lower 32 bits
% of register {\em rs1} into register {\em rd} (assembler pseudoinstruction
% SEXT.W).

\vspace{-0.4in}
\begin{center}
\begin{tabular}{R@{}W@{}R@{}R@{}R@{}R@{}O}
\\
\instbitrange{31}{26} &
\multicolumn{1}{c}{\instbit{25}} &
\instbitrange{24}{20} &
\instbitrange{19}{15} &
\instbitrange{14}{12} &
\instbitrange{11}{7} &
\instbitrange{6}{0} \\
\hline
\multicolumn{1}{|c|}{imm[11:6]} &
\multicolumn{1}{|c|}{imm[5]} &
\multicolumn{1}{|c|}{imm[4:0]} &
\multicolumn{1}{c|}{rs1} &
\multicolumn{1}{c|}{funct3} &
\multicolumn{1}{c|}{rd} &
\multicolumn{1}{c|}{opcode} \\
\hline
6 & \multicolumn{1}{c}{1} & 5 & 5 & 3 & 5 & 7 \\
000000 & shamt[5] & shamt[4:0]  & src & SLLI & dest & OP-IMM \\
000000 & shamt[5] & shamt[4:0]  & src & SRLI & dest & OP-IMM \\
010000 & shamt[5] & shamt[4:0]  & src & SRAI & dest & OP-IMM \\
000000 &       0  & shamt[4:0]  & src & SLLIW & dest & OP-IMM-32 \\
000000 &       0  & shamt[4:0]  & src & SRLIW & dest & OP-IMM-32 \\
010000 &       0  & shamt[4:0]  & src & SRAIW & dest & OP-IMM-32 \\
\end{tabular}
\end{center}

按常量移位被编码为一种专门化的I类型格式,它使用与RV32I相同的指令操作码。
对于RV64I,被移位的操作数在{\em rs1}中,移位的数目被编码在I立即数域的低6位中。
右移类型被编码在底第30位上。
SLLI是逻辑左移(移位后低位补零);SRLI是逻辑右移(移位后高位补零);而SRAI是算数右移(原始符号位被复制到空白的高位中)。
% Shifts by a constant are encoded as a specialization of the I-type
% format using the same instruction opcode as RV32I.  The operand to be
% shifted is in {\em rs1}, and the shift amount is encoded in the lower
% 6 bits of the I-immediate field for RV64I.  The right shift type is
% encoded in bit 30.  SLLI is a logical left shift (zeros are shifted
% into the lower bits); SRLI is a logical right shift (zeros are shifted
% into the upper bits); and SRAI is an arithmetic right shift (the
% original sign bit is copied into the vacated upper bits).

SLLIW、SRLIW和SRAIW是RV64I中独有的指令,它们的定义类似,但是在32位值上操作,并把它们的32位结果符号扩展到64位。
带有$imm[5] \neq 0$的SLLIW、SRLIW和SRAIW的编码是保留的。
% SLLIW, SRLIW, and SRAIW are RV64I-only instructions that are
% analogously defined but operate on 32-bit values and
% sign-extend their 32-bit results to 64 bits.
% SLLIW, SRLIW, and SRAIW encodings with $imm[5] \neq 0$ are reserved.

\begin{commentary}
  先前,$imm[5] \neq 0$的SLLIW、SRLIW和SRAIW被定义为:引发非法指令异常,而现在它们被标记为保留的。
  这是一个向后兼容的改变。
  % Previously, SLLIW, SRLIW, and SRAIW with $imm[5] \neq 0$ were defined to
  % cause illegal instruction exceptions, whereas now they are marked as
  % reserved.  This is a backwards-compatible change.
\end{commentary}

\vspace{-0.2in}
\begin{center}
\begin{tabular}{U@{}R@{}O}
\\
\instbitrange{31}{12} &
\instbitrange{11}{7} &
\instbitrange{6}{0} \\
\hline
\multicolumn{1}{|c|}{imm[31:12]} &
\multicolumn{1}{c|}{rd} &
\multicolumn{1}{c|}{opcode} \\
\hline
20 & 5 & 7 \\
U-立即数[31:12] & dest & LUI \\
U-立即数[31:12] & dest & AUIPC
\end{tabular}
\end{center}

LUI(加载高位立即数)使用与RV32I相同的操作码。
LUI把32位的U立即数放进寄存器{\em rd}中,并把最低的12位填充为零。该32位结果被符号扩展到64位。
% LUI (load upper immediate) uses the same opcode as RV32I.  LUI places
% the 32-bit U-immediate into register {\em rd}, filling in the lowest 12
% bits with zeros.
% The 32-bit result is sign-extended to 64 bits.

AUIPC(加高位立即数到{\tt pc})使用与RV32I相同的操作码。
AUIPC被用于构建关于{\tt pc}的相对地址,并使用U类型格式。
AUIPC从U立即数形成32位偏移量,并把最低的12位填充为零,把结果符号扩展到64位,把它加到AUIPC指令的地址,
然后把结果放进寄存器{\em rd}中。
% AUIPC (add upper immediate to {\tt pc}) uses the same opcode as RV32I.
% AUIPC is used to build {\tt
%   pc}-relative addresses and uses the U-type format.  AUIPC forms a 32-bit
% offset from the U-immediate, filling in the lowest 12 bits with zeros,
% sign-extends the result to 64 bits,
% adds it to the address of the AUIPC instruction,
% then places the result in register {\em rd}.

\begin{commentary}
注意,可以通过将LUI与LD配对、将AUIPC与JALR配对等方式形成的偏移量的地址集是[${-}2^{31}{-}2^{11}$, $2^{31}{-}2^{11}{-}1$]。
% Note that the set of address offsets that can be formed by pairing LUI
% with LD, AUIPC with JALR, etc.\@ in RV64I is
% [${-}2^{31}{-}2^{11}$, $2^{31}{-}2^{11}{-}1$].
\end{commentary}

\subsubsection*{整数寄存器—寄存器操作}

\vspace{-0.2in}
\begin{center}
\begin{tabular}{S@{}R@{}R@{}S@{}R@{}O}
\\
\instbitrange{31}{25} &
\instbitrange{24}{20} &
\instbitrange{19}{15} &
\instbitrange{14}{12} &
\instbitrange{11}{7} &
\instbitrange{6}{0} \\
\hline
\multicolumn{1}{|c|}{funct7} &
\multicolumn{1}{c|}{rs2} &
\multicolumn{1}{c|}{rs1} &
\multicolumn{1}{c|}{funct3} &
\multicolumn{1}{c|}{rd} &
\multicolumn{1}{c|}{opcode} \\
\hline
7 & 5 & 5 & 3 & 5 & 7 \\
0000000 & src2 & src1 & SLL/SRL     & dest & OP    \\
0100000 & src2 & src1 & SRA         & dest & OP    \\
0000000 & src2 & src1 & ADDW        & dest & OP-32    \\
0000000 & src2 & src1 & SLLW/SRLW   & dest & OP-32    \\
0100000 & src2 & src1 & SUBW/SRAW   & dest & OP-32    \\
\end{tabular}
\end{center}

ADDW和SUBW是RV64I独有的指令,它们的定义类似于ADD和SUB,但是在32位值上操作,并产生有符号的32位结果。
溢出被忽略,且结果的低32位被符号扩展到64位,并写到目的寄存器。
% ADDW and SUBW are RV64I-only instructions that are defined analogously
% to ADD and SUB but operate on 32-bit values and produce signed 32-bit
% results.  Overflows are ignored, and the low 32-bits of the result is
% sign-extended to 64-bits and written to the destination register.

SLL、SRL和SRA对寄存器{\em rs1}中的值实施逻辑左移、逻辑右移和算数右移,
移位的数目保持在寄存器{\em rs2}中。在RV64I中,只有{\em rs2}的低6位被考虑用于移位数目。
% SLL, SRL, and SRA perform logical left, logical right, and arithmetic
% right shifts on the value in register {\em rs1} by the shift amount
% held in register {\em rs2}.  In RV64I, only the low 6 bits of {\em
%   rs2} are considered for the shift amount.

SLLW、SRW和SRAW是RV64I独有的指令,它们的定义类似,但是在32位值上操作,并把它们的32位结果符号扩展到64位。
移位数量由{\em rs2[4:0]}给出。
% SLLW, SRLW, and SRAW are RV64I-only instructions that are analogously
% defined but operate on 32-bit values and
% sign-extend their 32-bit results to 64 bits.
% The shift amount is given by {\em rs2[4:0]}.

\section{加载和存储指令}

RV64I把地址空间扩展到了64位。执行环境将定义地址空间的哪部分对于访问是合法的。
% RV64I extends the address space to 64 bits.  The execution environment
% will define what portions of the address space are legal to access.

\vspace{-0.4in}
\begin{center}
\begin{tabular}{M@{}R@{}S@{}R@{}O}
\\
\instbitrange{31}{20} &
\instbitrange{19}{15} &
\instbitrange{14}{12} &
\instbitrange{11}{7} &
\instbitrange{6}{0} \\
\hline
\multicolumn{1}{|c|}{imm[11:0]} &
\multicolumn{1}{c|}{rs1} &
\multicolumn{1}{c|}{funct3} &
\multicolumn{1}{c|}{rd} &
\multicolumn{1}{c|}{opcode} \\
\hline
12 & 5 & 3 & 5 & 7 \\
offset[11:0] & base & width & dest & LOAD \\
\end{tabular}
\end{center}

\vspace{-0.2in}
\begin{center}
\begin{tabular}{O@{}R@{}R@{}S@{}R@{}O}
\\
\instbitrange{31}{25} &
\instbitrange{24}{20} &
\instbitrange{19}{15} &
\instbitrange{14}{12} &
\instbitrange{11}{7} &
\instbitrange{6}{0} \\
\hline
\multicolumn{1}{|c|}{imm[11:5]} &
\multicolumn{1}{c|}{rs2} &
\multicolumn{1}{c|}{rs1} &
\multicolumn{1}{c|}{funct3} &
\multicolumn{1}{c|}{imm[4:0]} &
\multicolumn{1}{c|}{opcode} \\
\hline
7 & 5 & 5 & 3 & 5 & 7 \\
offset[11:5] & src & base & width & offset[4:0] & STORE \\
\end{tabular}
\end{center}

对于RV64I,LD指令从内存加载一个64位的值到寄存器{\em rd}。
% The LD instruction loads a 64-bit value from memory into register {\em
%   rd} for RV64I.

对于RV64I,LW指令从内存加载一个32位的值,并把它符号扩展到64位,然后将其存储到寄存器{\em rd}。
另一方面,RV64I的LWU指令则会对内存中的32位值使用零扩展。
类似地,LH和LHU被定义用于16位值,以及LB和LBU用于8位值。
SD、SW、SH和SB指令分别把寄存器{\em rs2}的低64位、32位、16位和8位值存储到内存。
% The LW instruction loads a 32-bit value from memory and sign-extends
% this to 64 bits before storing it in register {\em rd} for RV64I.  The
% LWU instruction, on the other hand, zero-extends the 32-bit value from
% memory for RV64I.  LH and LHU are defined analogously for 16-bit
% values, as are LB and LBU for 8-bit values.  The SD, SW, SH, and SB
% instructions store 64-bit, 32-bit, 16-bit, and 8-bit values from the
% low bits of register {\em rs2} to memory respectively.

\section{“提示”指令}
\label{sec:rv64i-hints}

所有在RV32I中作为微架构HINT的指令(见~\ref{sec:rv32i-hints}节)也是RV64I中的HINT。
RV64I中的额外的运算指令同时扩展了标准HINT和自定义HINT的编码空间。
% All instructions that are microarchitectural HINTs in RV32I (see
% Section~\ref{sec:rv32i-hints}) are also HINTs in RV64I.  The
% additional computational instructions in RV64I expand both the standard and
% custom HINT encoding spaces.

表~\ref{tab:rv64i-hints}列出了所有的RV64I HINT代码点。91\%的HINT空间被保留用于标准HINT,
但是目前还没有被定义。其余的HINT空间被指定用于自定义HINT:不会有标准HINT将被定义在这个子空间中。
% Table~\ref{tab:rv64i-hints} lists all RV64I HINT code points.  91\% of the HINT
% space is reserved for standard HINTs.  The
% remainder of the HINT space is designated for custom HINTs: no standard HINTs
% will ever be defined in this subspace.

\begin{table}[hbt]
\centering
\begin{tabular}{|l|l|c|l|}
  \hline
  指令                  & 约束                                        & 代码点 & 目的\\ \hline \hline
  LUI                   & {\em rd}={\tt x0}                           & $2^{20}$                    & \multirow{11}{*}{\em 保留供未来标准使用} \\ \cline{1-3}
  AUIPC                 & {\em rd}={\tt x0}                           & $2^{20}$                    & \\ \cline{1-3}
  \multirow{2}{*}{ADDI} & {\em rd}={\tt x0}, 要么               & \multirow{2}{*}{$2^{17}-1$} & \\
                        & {\em rs1}$\neq${\tt x0} 要么 {\em imm}$\neq$0 &                             & \\ \cline{1-3}
  ANDI                  & {\em rd}={\tt x0}                           & $2^{17}$                    & \\ \cline{1-3}
  ORI                   & {\em rd}={\tt x0}                           & $2^{17}$                    & \\ \cline{1-3}
  XORI                  & {\em rd}={\tt x0}                           & $2^{17}$                    & \\ \cline{1-3}
  ADDIW                 & {\em rd}={\tt x0}                           & $2^{17}$                    & \\ \cline{1-3}
  ADD                   & {\em rd}={\tt x0}, {\em rs1}$\neq${\tt x0}  & $2^{10}-32$                 & \\ \cline{1-3}
  \multirow{2}{*}{ADD}  & {\em rd}={\tt x0}, {\em rs1}={\tt x0},      & \multirow{2}{*}{$28$}       & \\
                        & {\em rs2}$\neq${\tt x2}--{\tt x5}           &                             & \\ \hline
  \multirow{4}{*}{ADD}  & \multirow{4}{*}{\shortstack[l]{{\em rd}={\tt x0}, {\em rs1}={\tt x0}, \\{\em rs2}={\tt x2}--{\tt x5}}}
                                                                      & \multirow{4}{*}{$4$}        & ({\em rs2}={\tt x2}) NTL.P1 \\
                        &                                             &                             & ({\em rs2}={\tt x3}) NTL.PALL \\
                        &                                             &                             & ({\em rs2}={\tt x4}) NTL.S1 \\
                        &                                             &                             & ({\em rs2}={\tt x5}) NTL.ALL \\ \hline
  SUB                   & {\em rd}={\tt x0}                           & $2^{10}$                    & \multirow{22}{*}{\em 保留供未来标准使用} \\ \cline{1-3}
  AND                   & {\em rd}={\tt x0}                           & $2^{10}$                    & \\ \cline{1-3}
  OR                    & {\em rd}={\tt x0}                           & $2^{10}$                    & \\ \cline{1-3}
  XOR                   & {\em rd}={\tt x0}                           & $2^{10}$                    & \\ \cline{1-3}
  SLL                   & {\em rd}={\tt x0}                           & $2^{10}$                    & \\ \cline{1-3}
  SRL                   & {\em rd}={\tt x0}                           & $2^{10}$                    & \\ \cline{1-3}
  SRA                   & {\em rd}={\tt x0}                           & $2^{10}$                    & \\ \cline{1-3}
  ADDW                  & {\em rd}={\tt x0}                           & $2^{10}$                    & \\ \cline{1-3}
  SUBW                  & {\em rd}={\tt x0}                           & $2^{10}$                    & \\ \cline{1-3}
  SLLW                  & {\em rd}={\tt x0}                           & $2^{10}$                    & \\ \cline{1-3}
  SRLW                  & {\em rd}={\tt x0}                           & $2^{10}$                    & \\ \cline{1-3}
  SRAW                  & {\em rd}={\tt x0}                           & $2^{10}$                    & \\ \cline{1-3}
  \multirow{3}{*}{FENCE}& {\em rd}={\tt x0}, {\em rs1}$\neq${\tt x0}, & \multirow{3}{*}{$2^{10}-63$}& \\
                        & {\em fm}=0, 且 {\em pred}=0 或者 {\em succ}=0                      &                             & \\ \cline{1-3}
  %                      &                &                             & \\ 
  \multirow{3}{*}{FENCE}& {\em rd}$\neq${\tt x0}, {\em rs1}={\tt x0}, & \multirow{3}{*}{$2^{10}-63$}& \\
                        & {\em fm}=0, 且 {\em pred}=0 或者 {\em succ}=0                     &                             & \\ \cline{1-3}
  %                      &                 &                             & \\ 
  \multirow{2}{*}{FENCE}& {\em rd}={\em rs1}={\tt x0}, {\em fm}=0,    & \multirow{2}{*}{15}         & \\
                        & {\em pred}=0, {\em succ}$\neq$0             &                             & \\ \cline{1-3}
  \multirow{2}{*}{FENCE}& {\em rd}={\em rs1}={\tt x0}, {\em fm}=0,    & \multirow{2}{*}{15}         & \\
                        & {\em pred}$\neq$W, {\em succ}=0             &                             & \\ \hline
  \multirow{2}{*}{FENCE}& {\em rd}={\em rs1}={\tt x0}, {\em fm}=0,    & \multirow{2}{*}{1}          & \multirow{2}{*}{暂停} \\
                        & {\em pred}=W, {\em succ}=0                  &                             & \\ \hline
  SLTI                  & {\em rd}={\tt x0}                           & $2^{17}$                    & \multirow{10}{*}{\em 指定供自定义使用} \\ \cline{1-3}
  SLTIU                 & {\em rd}={\tt x0}                           & $2^{17}$                    & \\ \cline{1-3}
  SLLI                  & {\em rd}={\tt x0}                           & $2^{11}$                    & \\ \cline{1-3}
  SRLI                  & {\em rd}={\tt x0}                           & $2^{11}$                    & \\ \cline{1-3}
  SRAI                  & {\em rd}={\tt x0}                           & $2^{11}$                    & \\ \cline{1-3}
  SLLIW                 & {\em rd}={\tt x0}                           & $2^{10}$                    & \\ \cline{1-3}
  SRLIW                 & {\em rd}={\tt x0}                           & $2^{10}$                    & \\ \cline{1-3}
  SRAIW                 & {\em rd}={\tt x0}                           & $2^{10}$                    & \\ \cline{1-3}
  SLT                   & {\em rd}={\tt x0}                           & $2^{10}$                    & \\ \cline{1-3}
  SLTU                  & {\em rd}={\tt x0}                           & $2^{10}$                    & \\ \hline
\end{tabular}
\caption{RV64I “提示”指令}
\label{tab:rv64i-hints}
\end{table}
